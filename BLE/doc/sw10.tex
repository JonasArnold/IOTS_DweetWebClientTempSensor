\documentclass{article}
% Useful packages
\usepackage{xcolor}
\usepackage{amsmath}
\usepackage{graphicx}
\usepackage{listings}
\usepackage[colorlinks=true, allcolors=blue]{hyperref}

\definecolor{mGreen}{rgb}{0,0.6,0}
\definecolor{mGray}{rgb}{0.5,0.5,0.5}
\definecolor{mPurple}{rgb}{0.58,0,0.82}
\definecolor{backgroundColour}{rgb}{0.95,0.95,0.92}

\lstdefinestyle{CStyle}{
    backgroundcolor=\color{backgroundColour},   
    commentstyle=\color{mGreen},
    keywordstyle=\color{magenta},
    numberstyle=\tiny\color{mGray},
    stringstyle=\color{mPurple},
    basicstyle=\footnotesize,
    breakatwhitespace=false,         
    breaklines=true,                 
    captionpos=b,                    
    keepspaces=true,                 
    numbers=left,                    
    numbersep=5pt,                  
    showspaces=false,                
    showstringspaces=false,
    showtabs=false,                  
    tabsize=2,
    language=C
}

% Language setting
% Replace `english' with e.g. `spanish' to change the document language
\usepackage[english]{babel}

% Set page size and margins
% Replace `letterpaper' with`a4paper' for UK/EU standard size
\usepackage[letterpaper,top=2cm,bottom=2cm,left=3cm,right=3cm,marginparwidth=1.75cm]{geometry}



\title{IOTS BLE Central / Peripheral}
\author{Jonas Arnold, Simon Frei}

\begin{document}
\maketitle

\begin{abstract}
Auf dem Arduino MKR 1010 WiFi soll ein BLE Central und ein BLE Peripheral implementiert werden. Das Central soll vom Peripheral Daten lesen, ausserdem soll es mit einer custom Characteristic eine LED ansteuern können. 
\end{abstract}

\section{Implementierung BLE Central / Peripheral}

\subsection{Impelemtierung BLE Peripheral }

Die Implementierung des Peripherals lief problemlos ab. Die Spezifikationen des vordefinierten BLE Services \textbf{Environmental Sensing}. haben wir auf der \href{https://www.bluetooth.com/specifications/specs/}{Spezifikationenseite von Bluetooth} gefunden. Die vordefinierten UUIDs für den Service und die Characteristics waren ebenfalls bei \href{ bluetooth.com/specifications/assigned-numbers }{bluetooth.com} zu finden. 

Damit das ENV Board nicht zu oft abgefragt wird, wird die funktion \textbf{millis()} von Arduino verwendet und nur alle 1000 ms neue Sensordaten geschickt. Die LED kann auch schneller geschaltet werden (diese Characteristic wird im Loop ausserhalb dieser IF Abfrage bedient):
\begin{lstlisting}[style=CStyle, language=C, caption=Limitierung der Sensordatenabfragen]  % Start your code-block

// update sensors every SENSOR_UPDATE_TIME_MS to not stress the ENV board
if((millis() - last_update) > SENSOR_UPDATE_TIME_MS)
{
int16_t temperature = ENV.readTemperature();
int16_t humidity    = ENV.readHumidity();
int16_t illuminance = ENV.readIlluminance();

temperatureCharacteristic.writeValue(temperature);
humidityCharacteristic.writeValue(humidity);
illuminanceCharacteristic.writeValue(illuminance);

last_update = millis();  // update time
}

\end{lstlisting}

Bezüglich der Datentypen haben wir folgendes gelernt: Der Datentyp ist nicht an eine vordefinierte Characteristic verknüpft wie das in anderen Standards der Fall ist. Die Sensordaten würden bestenfalls als \textit{float} übermittelt (Datentyp der Characteristic \textit{BLEFloatCharacteristic}). Leider kann das die verwendete App \textbf{LightBlue} nicht richtig anzeigen (nur den HEX Wert davon). Deshalb haben wir schlussendlich die Werte als Integer übermittelt (Datentyp der Characteristic \textit{BLEIntCharacteristic}).
Dies kann die App richtig anzeigen, vorausgesetzt das Datenformat wird auf \textbf{Unsigned Little-Endian} eingestellt (Übermittlungsreihenfolge der einzelnen Bytes).

\subsection{Impelemtierung BLE Central  }


\end{document}